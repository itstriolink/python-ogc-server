\chapter{Installation}
The process of setting up this server is very simple. This is made possible by docker and docker compose, which makes it even easier to set-up and run. The application, in order to be integrated with a front-end application, you just need to point it to the URL of the server and make the API calls in conformace with the "OGC API - Features". \\
The server can also be hosted locally just by cloning the repository and using an ASGI server to run it.\\
\section{Cloning and Containerizing}
To checkout the source from GitLab, this command needs to be run:
\begin{minted}{bash}
$ git clone https://gitlab.com/labiangashi/python-wfs-server.git
\end{minted}
After that, the application can be hosted using docker and docker-compose. Before doing that, first, the \textit{docker-compose.yaml} file needs to be modified in order to include the correct environment variables for the \textit{collections} files. The environment variables can be modified here in the docker compose file: 
\begin{minted}{yaml}
version: '3'
services:
  ogcapi_server:
    build: .
    ports:
      - "80:8000"
    environment:
      - COLLECTIONS=castles=/app/wfs/osm-castles-CH.geojson
      - PORT=8000
\end{minted}
This docker compose file is a simple file which gives a name to the docker image, tells it what directory to build, what ports to expose and the environment variables required for the application.\\
\newline
After setting up the correct environment variables (the \textit{collections} and the port), the next step is building that docker image and then running it.
The Dockerfile already runs the server using uvicorn as soon as the image is run\\
This is how the Dockerfile looks:
\begin{minted}{yaml}
FROM tiangolo/uvicorn-gunicorn:python3.7-alpine3.8
WORKDIR /app/wfs
COPY requirements.txt /app/wfs/
RUN apk add build-base python-dev py-pip jpeg-dev zlib-dev
ENV LIBRARY_PATH=/lib:/usr/lib
RUN pip install -r requirements.txt
COPY . /app/wfs
EXPOSE 8000
CMD ["uvicorn", "ogc_api.main:app", "--host", "0.0.0.0", "--port", "8000"]
\end{minted}
\section{Running the server}
In order to build the image and then run it, the two following commands need to be executed in the terminal:
\begin{minted}{bash}
$ docker-compose build
\end{minted}
and
\begin{minted}{bash}
$ docker-compose up
\end{minted}
And that's it, the docker image with the running server is ready for production.
