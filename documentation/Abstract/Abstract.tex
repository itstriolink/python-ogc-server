This project serves as a "Mini Feature And Tile Server" which is minimally compliant with one of the geospatial standards defined in the web.\\
These standards define how APIs should be built \& developed in order to standardize the ways of retrieving geospatial data from the web using modern web development practices.\\
\newline
The standard that this project is minimally compliant with is the "\href{https://docs.opengeospatial.org/DRAFTS/17-069r1.html}{OGC API - Features}" standard. It is a standard created by the \href{https://www.opengeospatial.org/}{Open Geospatial Consortium (OGC)} which is an international consortium consisting of hundreds of institutions that encourage development and implementation of open standards for geospatial content and services. \\
The OGC API - Features standard specifies the behaviour of Web APIs that provide access to geospatial data. Various requests to the API enable the user to retrieve information about the underlying data set that it contains.\\
This application receives one or multiple GeoJSON files as input, pre-processes them and stores them as \textit{collections}, and then makes them available for querying and investigating to the API requesters.\\

The application is built using the Python programming language and is also dockerized, which makes it simple \& efficient to be setup in any host system.\\
\newline
Keywords: \textit{OGC, OGC API, GDAL, OAPIF, GeoJSON, API, Python, Docker.}
