\chapter{Introduction}
The documentation for this project is split into four main parts. 
The first chapter gives an introduction to the work done in relation with the goals and requirements that this project was set initially to meet. It also explains about the standards that this application follows and also some important notations related to those standards.\\
\newline
The second chapter has to do with the architecture and the design of the application. It shows some design decision, some of the technologies that were used, some useful information regarding those technologies and the reason why they were used for this project.\\
\newline
After the things above have been clarified, we now delve into the third chapter which explains the actual implementation of the application. It mostly contains the main notations about the project, what they mean and how they were used in the benefit of the application. It also explains about the code repository that was used and a little bit of how the project was documented.\\
\newline
The fourth and final chapter presents the results that were achieved from this project work and the outlook of it. It wraps up with a conclusion about the project and my personal opinions about this project.
\newpage
\section{Goals and Requirements}
The main objective for this project is creating a back-end application that is minimally compliant with the "OGC API - Features" \cite{OGCApiFeatures}. The application also has to support correct API responses for the OAPIF \cite{OAPIF} driver (which is an acronym for OGC API Features). It also has to support a couple more API endpoints that return raster tiles. More technical explanation about those endpoints and how they work will be written below and in the implementation chapter.\\
The main tasks include: returning GeoJSON objects given a specific collection, with the option of limiting the features to a specific bounding box within a region or even limiting the result set to as many features as the request prefers. These are the tasks that also should be compliant with the "OGC API - Features".\\
\newline
This project not only serves as an API that is minimally compliant with the "OGC API - Features" standard, but it also serves as a possible replacement for an application developed by the \href{https://www.hsr.ch/geometalab}{HSR Geometa Lab}. This application, called \href{https://castle-map.infs.ch/}{Castle Map} is an application that shows all the castles in Switzerland within a map layer.\\
This application utilizes the API's methods to return all the features within a specific area (in this case, Switzerland) and adds links to those features from which users can access the \href{https://www.wikidata.org/wiki/Wikidata:Main_Page}{Wikidata} about that feature, the Wikipedia data or even open it on \href{https://www.openstreetmap.org/node/1500554009}{OpenStreetMap}. It also uses some of the other API endpoints that this API offers in order to server map tiles (which will be explained more thoroughly below) as images on top of the map layer that displays the features, or in this case, the castles.\\
The features are sorted based on the views the castles receive (in Wikidata) and then are displayed in a sort-of prioritized way in the map layer.\\
The Castle Map application contains an MIT license and is contained here:\\ \href{https://gitlab.com/geometalab/castle-map}{gitlab.com/geometalab/castle-map}\\
\newline
Another requirement for this application is to be operable by the \href{https://gdal.org/programs/ogrinfo.html}{ogrinfo} program, which is a program contained in the "GDAL" library (more details about GDAL below) that lists information about a data source that is supported by GDAL.\\
In this case, the ogrinfo program must be able to make requests to this application and receive correct responses from it using the OAPIF driver, which is a driver for connecting to OGC API servers. 
\newpage

\section{Open Geospatial Consortium API - Features}
\href{https://www.opengeospatial.org/}{Open Geospatial Consortium (OGC)} is an international consortium consisting of hundreds of institutions that encourage development and implementation of open standards for geospatial content and services.\\
OGC API - Features is a standard, created by OGC that is very important for this project since this project implements it's features and tasks the standard defined \href{https://docs.opengeospatial.org/is/17-069r3/17-069r3.html}{HERE}.

Portion of text directly taken from the OGC API - Features Abstract:  \cite{OGCApiFeatures}\\
"OGC API standards define modular API building blocks to spatially enable Web APIs in a consistent way. The OpenAPI specification is used to define the API building blocks.

The OGC API family of standards is organized by resource type. This standard specifies the fundamental API building blocks for interacting with features. The spatial data community uses the term 'feature' for things in the real world that are of interest.

OGC API Features provides API building blocks to create, modify and query features on the Web. OGC API Features is comprised of multiple parts, each of them is a separate standard. This part, the "Core", specifies the core capabilities and is restricted to fetching features where geometries are represented in the coordinate reference system WGS 84 with axis order longitude/latitude." \cite{OGCApiFeatures}

\section{Geospatial Data Abstraction Library}
Geospatial Data Abstraction library or GDAL is a very important notation for this project since a lot of the drivers and features in their library can be used in conjunction with this project to deliver geospatial data and different responses.\\
\newline
GDAL is a translator library for raster and vector geospatial data formats that is released under an X/MIT style Open Source License by the Open Source Geospatial Foundation. As a library, it presents a single raster abstract data model and single vector abstract data model to the calling application for all supported formats. It also comes with a variety of useful command line utilities for data translation and processing. \cite{WhatIsGDAL}\\
GDAL also provides a lot of drivers, and the driver that is important for this project is the \href{https://gdal.org/drivers/vector/oapif.html}{OAPIF} driver (Short for OGC API - Features).\\
This driver provides us the opportunity to query OGC API servers and to retrieve Geo information from those APIs.

\section{Tiled Web Map}
A tiled web map is a map that is displayed is a map that is displayed by joining a lot of individually requested images or vector files over the internet. It is currently the most popular way to display maps, which replaces the old methods such as \href{https://www.opengeospatial.org/standards/wms}{Web Map Service (WMS)}, which usually displayed a single but large map, that was navigable using different arrow buttons. The first technology for displaying tiles used raster tiles and then after that, vector tiles were introduced.\\
The reason why this standard is important for this project is that this application implements some of the methods that are mentioned there.\\
\subsection{XYZ Specification}
One of the most popular ways to serve these tiles is using the XYZ specification, Google was one of the first companies to majorly use this way of serving tiles. This request, then usually follows the de facto OpenStreetMap standard (known as Slippy Map \cite{WhatIsSlippyMap}) for tiles that follows these rules:
\begin{itemize}
\item Tiles are 256 × 256 pixel PNG files
\item Each zoom level is a directory, each column is a subdirectory, and each tile in that column is a file
\item The request URL looks like \textit{http://.../\{z\}/\{x\}/\{y\}.png}, where \textit{Z} is the zoom level, and \textit{X} and \textit{Y} identify the tile's position. 
\end{itemize}
