\chapter{Results}

\section{Achievements}
These were/are the goals for this project that have all been met.\\
\begin{itemize}
	\item Implement a 'Mini-Feature and Tile Server' in Python.
	\begin{itemize}
		\item Input big, sorted GeoJSON files
		\item Preprocess and format those files containing features and return them to the HTTP response in the appropriate format (using the OGC API - Standards format), support limiting the results in a given bounding box.
		\item Return a specific feature in the feature collection, it's metadata and it's bounding box.
		\item Return a raster tile (256x256) which is then displayed on top of a map to display where the features are located in the map (used in conjunction with \href{maptile.com}{maptile} for example).
		\item Dockerize the application so that it can easily be shipped between different hosts.
	\end{itemize}
	\item Integrate this in the fork of \href{https://gitlab.com/geometalab/castle-map}{Castle Dossier Map}
	\item Integration tests with OAPIF/ogrinfo
\end{itemize}
There was another thing which was a goal in the beginning of the project, but hasn't been implemented because it didn't seem useful for now.\\
That goal was: Integration-Tests of MFTS in Apache Superset (which is another thesis project).
\newpage
\section{Outlook}
The current version of this application is pretty robust and can be used to run a WFS/OGC client in tandem with.\\
It implements the OGC - API Features only minimally, there are some optional API endpoints that are mentioned in the standards but they have not been implemented here because the goal for this project was to only minimally support those standards, which means, only the required endpoints and rules.\\
\newline
Something that would maybe improve the application is to introduce async methods using the FastAPI framework, but that is not necessary for now.\\
Another thing would be implementing all the API endpoints (optional ones too) from the OGC - API Features standard but that is as well not needed now.
\section{Reflection}
This project was a great experience for me and taught me a lot of things that I had never heard of!\\
First off, GeoJSON, GDAL, OGC, QGIS, these were all completely unknown terms for me, thanks to this project, I now know what they are and how important they are for geomatics.\\
\newline
Another thing that was very new for me was Docker, containers is something I have heard of before, but never really used. This is the first time I have used Docker and containers to "ship" an application, even though it is a single-container application, I still learned a lot of new things from it.\\
\newline
And the most important part of this, is that the whole project is written in Python.\\
This was my first Python application and learning it was a big learning curve for me. I never worked with such syntax and such programming paradigm. I am mostly used to high level OOP programming languages like Java or C\#. So this was a completely new, exciting but sometimes even confusing for me.\\
\newline
Working with such technologies, tools, libraries and standards was all around a new and great experience for me, although frustrating at times, it definitely taught me a lot of new things!